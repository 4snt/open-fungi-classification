\documentclass{article}
\usepackage[utf8]{inputenc}
\usepackage{graphicx}
\usepackage{hyperref}
\title{Open Fungi Image Classification - Guia de Treinamento}
\author{Lord Python}
\date{Dezembro 2024}

\begin{document}

\maketitle

\section{Introdu\c{c}\~ao}
Este projeto \'{e} uma solu\c{c}\~ao para a classifica\c{c}\~ao de imagens microsc\'opicas de fungos utilizando 	extbf{Transfer Learning} com o modelo 	extbf{VGG16}. Ele abrange etapas como pr\'e-processamento de dados, treinamento do modelo, avalia\c{c}\~ao e visualiza\c{c}\~ao dos resultados.

\section{Funcionalidades}
\begin{itemize}
    \item 	extbf{Visualiza\c{c}\~ao do Dataset}: Exibe imagens por classe para compreens\~ao inicial dos dados.
    \item 	extbf{Transfer Learning}: Utiliza a arquitetura pr\'e-treinada VGG16.
    \item 	extbf{Grad-CAM}: Gera\c{c}\~ao de heatmaps para interpretar as previs\~oes do modelo.
    \item 	extbf{M\'etricas de Avalia\c{c}\~ao}: Matriz de confus\~ao e relat\'orio de classifica\c{c}\~ao.
    \item 	extbf{Treinamento Incremental}: Suporte para fine-tuning das camadas finais da VGG16.
\end{itemize}

\section{Estrutura do Projeto}
\begin{verbatim}
project/
|
├── data/
│   ├──fungi-images/
│   │    ├──test
|   |    ├──train
|   |    ├──valid  
│   │
│   ├── data_preprocessing.py       # Geradores de dados com aumento de dados
│   ├── data_visualization.py       # Visualiza\c{c}\~ao inicial do dataset
|
├── model/
│   ├── model_definition.py         # Defini\c{c}\~ao do modelo de Transfer Learning
│   ├── model_training.py           # L\'ogica de treinamento do modelo
│   ├── model_evaluation.py         # Avalia\c{c}\~ao e visualiza\c{c}\~ao dos resultados
|
├── utils/
│   ├── grad_cam.py                 # Fun\c{c}\~ao para Grad-CAM (visualiza\c{c}\~ao de heatmaps)
|
├── index.py                        # Script principal para gerenciar o fluxo do projeto
├── installer.py                    # Instalador autom\'atico de depend\^encias
├── requirements.txt                # Lista de depend\^encias do projeto
└── README.md                       # Documenta\c{c}\~ao do projeto
\end{verbatim}

\section{Pr\'e-requisitos}
Antes de iniciar, voc\^e precisa ter:
\begin{itemize}
    \item 	extbf{Python 3.7+}
    \item 	extbf{Pip} (gerenciador de pacotes do Python)
    \item As bibliotecas listadas no arquivo 	exttt{requirements.txt}
\end{itemize}

\section{Como Configurar o Projeto}
\begin{enumerate}
    \item Clone este reposit\'orio ou baixe os arquivos:
    \begin{verbatim}
    git clone https://github.com/seu-usuario/microscopic-fungi-classification.git
    cd microscopic-fungi-classification
    \end{verbatim}
    \item Instale as depend\^encias usando o instalador autom\'atico:
    \begin{verbatim}
    python installer.py
    \end{verbatim}
    \item Certifique-se de que o dataset est\'a organizado nos diret\'orios:
    \begin{verbatim}
    /kaggle/input/microscopic-fungi-images/
    ├── train/
    │   ├── Class1/
    │   ├── Class2/
    │   ...
    ├── valid/
    ├── test/
    \end{verbatim}
\end{enumerate}

\section{Como Executar}
\begin{enumerate}
    \item Para visualizar as imagens e treinar o modelo, execute o arquivo principal:
    \begin{verbatim}
    python index.py
    \end{verbatim}
    \item O script realizar\'a as seguintes etapas:
    \begin{itemize}
        \item Carregar\'a os dados do conjunto de treinamento, valida\c{c}\~ao e teste.
        \item Visualizar\'a as imagens por classe.
        \item Construir\'a o modelo utilizando Transfer Learning.
        \item Treinar\'a o modelo e avaliar\'a o desempenho.
        \item Gerar\'a relat\'orios e gr\'aficos de avalia\c{c}\~ao.
    \end{itemize}
\end{enumerate}

\section{Resultados}
\subsection{Gr\'aficos de Treinamento}
Exibe a precis\~ao e a perda durante o treinamento e valida\c{c}\~ao.

\subsection{Matriz de Confus\~ao}
Analisa os erros de classifica\c{c}\~ao do modelo.

\subsection{Grad-CAM Heatmaps}
Mostra as regi\~oes importantes nas imagens para cada classe predita.

\subsection{Relat\'orio de Classifica\c{c}\~ao}
Fornece m\'etricas detalhadas como precis\~ao, recall e F1-Score.

\section{Contribuindo}
Contribui\c{c}\~oes s\~ao bem-vindas! Siga os passos abaixo para colaborar:
\begin{enumerate}
    \item Fa\c{c}a um fork do reposit\'orio.
    \item Crie um branch para sua feature ou corre\c{c}\~ao de bug: 	exttt{git checkout -b minha-feature}.
    \item Fa\c{c}a o commit das suas altera\c{c}\~oes: 	exttt{git commit -m 'Adiciona nova funcionalidade'}.
    \item Envie para o branch principal: 	exttt{git push origin minha-feature}.
    \item Abra um Pull Request.
\end{enumerate}

\section{Licen\c{c}a}
Este projeto est\'a sob a licen\c{c}a MIT. Sinta-se \`a vontade para utiliz\'a-lo e modific\'a-lo conforme necess\'ario.

\end{document}